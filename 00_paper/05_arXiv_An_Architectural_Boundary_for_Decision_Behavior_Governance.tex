% Options for packages loaded elsewhere
\PassOptionsToPackage{unicode}{hyperref}
\PassOptionsToPackage{hyphens}{url}
%
\documentclass[
]{article}
\usepackage{amsmath,amssymb}
\usepackage{lmodern}
\usepackage{iftex}
\ifPDFTeX
  \usepackage[T1]{fontenc}
  \usepackage[utf8]{inputenc}
  \usepackage{textcomp} % provide euro and other symbols
\else % if luatex or xetex
  \usepackage{unicode-math}
  \defaultfontfeatures{Scale=MatchLowercase}
  \defaultfontfeatures[\rmfamily]{Ligatures=TeX,Scale=1}
\fi
% Use upquote if available, for straight quotes in verbatim environments
\IfFileExists{upquote.sty}{\usepackage{upquote}}{}
\IfFileExists{microtype.sty}{% use microtype if available
  \usepackage[]{microtype}
  \UseMicrotypeSet[protrusion]{basicmath} % disable protrusion for tt fonts
}{}
\makeatletter
\@ifundefined{KOMAClassName}{% if non-KOMA class
  \IfFileExists{parskip.sty}{%
    \usepackage{parskip}
  }{% else
    \setlength{\parindent}{0pt}
    \setlength{\parskip}{6pt plus 2pt minus 1pt}}
}{% if KOMA class
  \KOMAoptions{parskip=half}}
\makeatother
\usepackage{xcolor}
\setlength{\emergencystretch}{3em} % prevent overfull lines
\providecommand{\tightlist}{%
  \setlength{\itemsep}{0pt}\setlength{\parskip}{0pt}}
\setcounter{secnumdepth}{-\maxdimen} % remove section numbering
\ifLuaTeX
  \usepackage{selnolig}  % disable illegal ligatures
\fi
\IfFileExists{bookmark.sty}{\usepackage{bookmark}}{\usepackage{hyperref}}
\IfFileExists{xurl.sty}{\usepackage{xurl}}{} % add URL line breaks if available
\urlstyle{same} % disable monospaced font for URLs
\hypersetup{
  hidelinks,
  pdfcreator={LaTeX via pandoc}}

\author{}
\date{}

\begin{document}

\hypertarget{an-architectural-boundary-for-decision-behavior-governance-decision-constraint-layer-governance-properties-and-pdca-alignment}{%
\section{An Architectural Boundary for Decision Behavior Governance:
Decision Constraint Layer, Governance Properties, and PDCA
Alignment}\label{an-architectural-boundary-for-decision-behavior-governance-decision-constraint-layer-governance-properties-and-pdca-alignment}}

\textbf{Author:} Spark Tsai \textbf{Date:} January 2026
\textbf{Keywords:} AI Governance, Decision Constraint Layer (DCL),
ISO/IEC 42001, AISCL, Evidence Sovereignty, PDCA.

\begin{center}\rule{0.5\linewidth}{0.5pt}\end{center}

\hypertarget{abstract}{%
\subsection{Abstract}\label{abstract}}

In the era of foundation models deployed via commercial APIs,
organizations no longer possess control over internal model reasoning,
version semantics, or behavioral consistency. This loss of model
sovereignty creates a structural governance problem: how can accountable
AI governance be realized without model ownership? This paper introduces
the \textbf{Decision Constraint Layer (DCL)} as an explicit
architectural boundary for \textbf{Decision Behavior Governance}.
Positioned between decision intent and action execution, the DCL serves
as the non-bypassable location where organizational governance is
enforced, evidenced, and attributed. We define the structural conditions
required for governability---governance properties and carriers---and
demonstrate how these elements align with the \textbf{PDCA cycle}
mandated by \textbf{ISO/IEC 42001}, eenabling auditable AI governance as
an architectural property, rather than a post-hoc procedural claim.

\begin{center}\rule{0.5\linewidth}{0.5pt}\end{center}

\hypertarget{introduction-from-conceptual-governance-to-architectural-realization}{%
\subsection{1. Introduction: From Conceptual Governance to Architectural
Realization}\label{introduction-from-conceptual-governance-to-architectural-realization}}

This paper builds upon prior work defining Decision Behavior as the
governance target and identifying governance nullity in foundation model
environments.

As AI systems increasingly participate in organizational
decision-making, governance discussions have shifted toward
accountability and responsibility. However, acknowledging decision
behavior as a governance target does not automatically resolve how
governance is realized in practice.

In black-box model environments, governance mechanisms relying on model
introspection or post-hoc monitoring cannot guarantee that governance
conditions were enforced during decision formation. Without a clearly
defined architectural intervention point, governance remains declarative
rather than operative. This paper contends that AI governance requires a
non-bypassable architectural boundary---the \textbf{Decision Constraint
Layer (DCL)}---to transform governance from a procedural assertion into
an engineering fact.

\hypertarget{architectural-requirements-for-decision-behavior-governance}{%
\subsection{2. Architectural Requirements for Decision Behavior
Governance}\label{architectural-requirements-for-decision-behavior-governance}}

To govern decision behavior meaningfully, three requirements must be
satisfied:

\begin{enumerate}
\def\labelenumi{\arabic{enumi}.}
\tightlist
\item
  \textbf{Ex-ante Intervention:} Governance must intervene during
  decision formation, not merely observe outcomes.
\item
  \textbf{Structural Decoupling:} Governance must have a clearly defined
  boundary independent of the model engine.
\item
  \textbf{Deterministic Evidence:} Governance must be provable through
  traceable residues that establish Evidence Sovereignty.
\end{enumerate}

\hypertarget{the-decision-constraint-layer-dcl}{%
\subsection{3. The Decision Constraint Layer
(DCL)}\label{the-decision-constraint-layer-dcl}}

\hypertarget{architectural-positioning}{%
\subsubsection{3.1 Architectural
Positioning}\label{architectural-positioning}}

The DCL is a structural boundary located between \textbf{Decision
Intent} and \textbf{Action Execution}. It is the sole location
authorized to mediate decision behavior through enforceable constraints.

\begin{itemize}
\tightlist
\item
  \textbf{Mediation:} Ensuring behavior adheres to organizational
  premises.
\item
  \textbf{Attribution:} Establishing a causal link between institutional
  rules and AI outcomes.
\end{itemize}

\hypertarget{boundary-by-exclusion-non-goals}{%
\subsubsection{3.2 Boundary by Exclusion
(Non-Goals)}\label{boundary-by-exclusion-non-goals}}

To maintain integrity, the DCL explicitly does \textbf{NOT} perform:

\begin{itemize}
\tightlist
\item
  Prompt optimization or reasoning enhancement.
\item
  Model inference or training.
\item
  Post-hoc logging without prior constraint mapping.
\end{itemize}

These exclusions ensure that the DCL remains a governance boundary
rather than a reasoning enhancement mechanism.

\hypertarget{governance-properties-conditions-for-governability}{%
\subsection{4. Governance Properties: Conditions for
Governability}\label{governance-properties-conditions-for-governability}}

Before a governance cycle (e.g., PDCA) can operate, governance elements
must be governable. This paper defines five atomic properties:

\begin{enumerate}
\def\labelenumi{\arabic{enumi}.}
\tightlist
\item
  \textbf{Identity:} Unique identification of each governance element.
\item
  \textbf{Versioning:} Supporting evolution and comparison across time.
\item
  \textbf{Applicability Boundary:} Determinable trigger conditions.
\item
  \textbf{Risk Attribution:} Mapping elements to identifiable impacts.
\item
  \textbf{Authority:} Explicit institutional source of ownership.
\end{enumerate}

\hypertarget{governance-carriers-operationalizing-properties}{%
\subsection{5. Governance Carriers: Operationalizing
Properties}\label{governance-carriers-operationalizing-properties}}

Governance properties are embodied through three carriers:

\begin{itemize}
\tightlist
\item
  \textbf{Constraint Carrier:} Limits the admissible decision space
  (e.g., AISCL).
\item
  \textbf{Trigger Evidence Carrier:} Records how constraints were
  invoked during execution.
\item
  \textbf{Effective Artifact Carrier:} Represents the finalized outcome
  with traceable governance metadata.
\end{itemize}

\hypertarget{carrier-utilization-across-the-decision-lifecycle}{%
\subsection{6. Carrier Utilization across the Decision
Lifecycle}\label{carrier-utilization-across-the-decision-lifecycle}}

Governance is established only when carriers are actively utilized
during decision formation:

\begin{itemize}
\tightlist
\item
  \textbf{Premise Establishment:} Deployment of Constraint Carriers.
\item
  \textbf{Decision Formation:} Integration of Constraint and Trigger
  Evidence.
\item
  \textbf{Post-decision Review:} Analysis of Trigger Evidence and
  Artifacts.
\end{itemize}

\hypertarget{alignment-with-isoiec-42001-pdca}{%
\subsection{7. Alignment with ISO/IEC 42001
(PDCA)}\label{alignment-with-isoiec-42001-pdca}}

The DCL architecture provides a direct technical correspondence to the
ISO/IEC 42001 PDCA cycle:

\begin{itemize}
\tightlist
\item
  \textbf{Plan:} Defining Constraint Properties and AISCL rules.
\item
  \textbf{Do:} Constraint-mediated decision formation within the DCL.
\item
  \textbf{Check:} Aggregating Trigger Evidence to detect behavioral
  drift.
\item
  \textbf{Act:} Issuing versioned governance updates based on evidence
  analysis.
\end{itemize}

\hypertarget{evidence-sovereignty-and-institutional-learning}{%
\subsection{8. Evidence Sovereignty and Institutional
Learning}\label{evidence-sovereignty-and-institutional-learning}}

The DCL ensures that while model inference may remain probabilistic, the
resulting governance evidence is deterministic by construction.
Aggregated evidence allows organizations to detect when a model's
interpretation of constraints fails, enabling the \textbf{Act} phase of
PDCA to be a versioned governance action informed by data.

\hypertarget{conclusion}{%
\subsection{9. Conclusion}\label{conclusion}}

The Decision Constraint Layer provides an architectural foundation for
accountable AI governance without requiring model ownership. By defining
explicit boundaries and evidence-driven cycles, AI governance transcends
static documentation and becomes a verifiable engineering reality.

\end{document}
